\documentclass[a4paper,twoside,12pt]{article}
%\usepackage [reqno] {amsmath}
\usepackage{amsfonts,amstext}
\usepackage{amsmath}
\usepackage{german}
\usepackage{graphicx}
\usepackage{fullpage}

\newcommand{\ZETTELNUMMER}{5}

\newcounter{AUFGNR}
\setcounter{AUFGNR}{1}
\newcommand{\AUFGABE}[2]{\vspace{0.3cm}\item[Aufgabe~\arabic{AUFGNR}]\stepcounter{AUFGNR} #1\hfill\emph{#2}}


\newcommand{\floor}[1]{\left\lfloor{#1}\right\rfloor}
\newcommand{\ceil}[1]{\left\lceil{#1}\right\rceil}
\newcommand{\half}[1]{\frac{#1}{2}}
\newcommand{\N}{\mathbb{N}}


\renewcommand{\labelenumi}{(\alph{enumi})}
\renewcommand{\labelenumii}{(\roman{enumii})}


\begin{document}
\pagestyle{empty}
\hrule\medskip
\rule{0ex}{0ex}\\[-1ex]
\ZETTELNUMMER. Aufgabenblatt zur Vorlesung

\smallskip
\noindent
\large
\textbf{Nichtsequentielle und Verteilte Programmierung}\hfill SoSe
2018 \\[0.5ex]
\normalsize
Anton Oehler, Mark Niehues

\newcommand{\immer}{\Box}
\newcommand{\irgendwann}{\lozenge}
\newcommand{\folgt}{\Rightarrow}
\newcommand{\oder}{\vee}
\newcommand{\und}{\wedge}

\medskip\hrule

\begin{description}
% Aufgabe 1
\AUFGABE{Lineare Temporale Logik I}{10 Punkte}
\end{description}
\end{document}