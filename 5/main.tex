% /////////////////////// SETUP /////////////////////////


\documentclass[10pt,	
	parskip=half-,
	paper=a4,
	english,
	ngerman]{scrartcl}
% //////////////////// Pakete ////////////////////
\usepackage{amsmath, mathtools, amssymb, bm}
\usepackage{fontspec, microtype, lmodern}
\usepackage[cmyk,table]{xcolor}
\usepackage{tikz, tikz-qtree, fp, pgfplots, pgfplotstable, graphicx, pgf-umlcd}
%\usepackage{lscape}
\usepackage[normalem]{ulem} % hingefuegt von YC fuer \dashuline
\usepackage{listings, tabularx, url, xspace, float, ifthen, longtable, fullpage, multirow, cancel,  comment}
\usepackage[ngerman]{babel}
\usepackage[hidelinks]{hyperref}
\usepackage[shortlabels]{enumitem}
\usepackage{amsthm}

% //////////////////// Bibliotheken ////////////////////

%\usetikzlibrary{tikzmark, positioning, automata, arrows, shapes}
%\usetikzlibrary{calc, angles, quotes, babel} 				% Nadja
%\usetikzlibrary{decorations.pathmorphing}
%\usetikzlibrary{decorations.pathreplacing}
%\usetikzlibrary{decorations.shapes}
%\usetikzlibrary{decorations.text}

\usetikzlibrary{shadows} % hingefuegt von YC fuer ER-Modell


% //////////////////// Code ////////////////////
\lstloadlanguages{Python, Haskell, [LaTeX]TeX, Java, SQL, HTML}
\lstset{
	basicstyle=\footnotesize\ttfamily,						% \scriptsize the size of the fonts that are used for the code
	backgroundcolor = \color{bgcolour},					% legt Farbe der Box fest
	breakatwhitespace=false,								% sets if automatic breaks should only happen at whitespace
	breaklines=true,									% sets automatic line breaking
	captionpos=t,										% sets the caption-position to bottom, t for top
	commentstyle=\color{codeblue}\ttfamily,					% comment style
	frame=single,										% adds a frame around the code
	keepspaces=true,									% keeps spaces in text, useful for keeping indentation of code (possibly needs columns=flexible)
	keywordstyle=\bfseries\ttfamily\color{codepurple},		% keyword style
	numbers=left,										% where to put the line-numbers; possible values are (none, left, right)
	numberstyle=\tiny\color{codegreen},					% the style that is used for the line-numbers
	numbersep=7pt,										% how far the line-numbers are from the code
	stepnumber=1,										% nummeriert nur jede i-te Zeile
	showspaces=false,									% show spaces everywhere adding particular underscores; it overrides 'showstringspaces'
	showstringspaces=false,								% underline spaces within strings only
	showtabs=false,									% show tabs within strings adding particular underscores
	flexiblecolumns=false,
	tabsize=3,										% Tabgröße
	stringstyle=\color{orange}\ttfamily,					% string literal style
	numberblanklines=false,								% leere Zeilen werden nicht mitnummeriert
	xleftmargin=1.2em,									% Abstand zum linken Layoutrand
	xrightmargin=0.4em,									% Abstand zum rechten Layoutrand
	%aboveskip=2ex, 
}
\lstdefinestyle{py}{
	language=Python,
}
\lstdefinestyle{hs}{
	language=Haskell,
}
\lstdefinestyle{tex}{
	language=[LaTeX]TeX,
	escapeinside={\%*}{*)},								% if you want to add LaTeX within your code
	texcsstyle=*\bfseries\color{blue},						% hervorhebung der tex-Schlüsselwörter
	morekeywords={*,$,\{,\},\[,\],lstinputlisting,includegraphics,
		rowcolor,columncolor,listoffigures,lstlistoflistings,
		subsection,subsubsection,textcolor,tableofcontents,colorbox,
		fcolorbox,definecolor,cellcolor,url,linktocpage,subtitle,
		subject,maketitle,usetikzlibrary,node,path,addbibresource,
		printbibliography},								% if you want to add more keywords to the set
	numbers=none,
	numbersep=0pt,
	xleftmargin=0.4em,
}
\lstdefinestyle{java}{
	language=Java,
	extendedchars=true,									% lets you use non-ASCII characters; for 8-bits encodings only, does not work with UTF-8
}
\lstdefinelanguage[x64]{Assembler}							% add a "x64" dialect of Assembler
[x86masm]{Assembler}									% based on the "x86masm" dialect with these extra keywords:
	{morekeywords={CDQE,CQO,CMPSQ,CMPXCHG16B,JRCXZ,LODSQ,MOVSXD,%
		POPFQ,PUSHFQ,SCASQ,STOSQ,IRETQ,RDTSCP,SWAPGS,%
		rax,rdx,rcx,rbx,rsi,rdi,rsp,rbp,%
		r8,r8d,r8w,r8b,r9,r9d,r9w,r9b}
}													% for 8-bits encodings only, does not work with UTF-8
\lstdefinestyle{c}{
	language=c,
	extendedchars=true,									% for 8-bits encodings only, does not work with UTF-8
}
\lstdefinestyle{sql}{
	language=SQL,
}
\lstdefinestyle{html}{
	language=HTML,
}

% //////////////////// Commands ////////////////////
\newcommand\N{\mathbb{N}\xspace}
\newcommand\Q{\mathbb{Q}\xspace}
\newcommand\R{\mathbb{R}}
\newcommand\Z{\mathbb{Z}\xspace}
\newcommand\zz{\ensuremath{\raisebox{+0.25ex}{Z}				% zu zeigen
	\kern-0.4em\raisebox{-0.25ex}{Z}%
	\;\xspace}}

\DeclarePairedDelimiter\abs{\lvert}{\rvert}%					% Nadja
\DeclarePairedDelimiter{\ceil}{\lceil}{\rceil}%				% Nadja
\DeclarePairedDelimiter{\floor}{\lfloor}{\rfloor}%			% Nadja
\DeclarePairedDelimiter\ub{\mathcal{O}(}{)}%					% Nadja
\DeclarePairedDelimiter\lb{\Omega(}{)}%						% Nadja
\DeclarePairedDelimiter\eb{\Theta(}{)}%						% Nadja
\DeclarePairedDelimiter{\dx}{[}{]'}%            				% Nadja

\makeatletter											% Nadja
\newcommand{\subalign}[1]{%
	\vcenter{%
		\Let@ \restore@math@cr \default@tag
		\baselineskip\fontdimen10 \scriptfont\tw@
		\advance\baselineskip\fontdimen12 \scriptfont\tw@
		\lineskip\thr@@\fontdimen8 \scriptfont\thr@@
		\lineskiplimit\lineskip
		\ialign{\hfil$\m@th\scriptstyle##$&$\m@th\scriptstyle{}##$\crcr#1\crcr
		}%
	}
}
\makeatother

\newcolumntype{R}{>{\raggedleft\arraybackslash}p{2cm}}			% Nadja

\DeclareMathOperator{\ggT}{ggT}         					% Nadja

% //////////////////// Theorems ////////////////////
%\newtheorem{theorem}{Satz}
%\newtheorem{corollary}[theorem]{Folgerung}
%\newtheorem{lemma}[theorem]{Lemma}
%\newtheorem{observation}[theorem]{Beobachtung}
%\newtheorem{definition}[theorem]{Definition}
%\newtheorem{Literatur}[theorem]{Literatur}

%\makeatletter   % konfiguriert proof
%\newenvironment{Proof}[1][\proofname]{\par
%	\pushQED{\qed}%
%	\normalfont \topsep6\p@\@plus6\p@\relax
%	\trivlist
%	\item[\hskip\labelsep
%		\bfseries
%	#1\@addpunct{.}]\ignorespaces
%}{%
%	\popQED\endtrivlist\@endpefalse
%}
%\makeatother

% ////////////// Header //////////////
\newcommand{\head}[0]{%
	% Notenspiegel
	\vspace*{-15ex}									% rückt Logo an den oberen Seitenrand
	\makebox[\dimexpr\textwidth][l]{						% linksbündig
		\begin{minipage}{0.5\linewidth}
			\newcounter{tasks}
			\setcounter{tasks}{\taskCount}
			\stepcounter{tasks}							% exerciseCounter++
			\newcounter{no}
			\def\and{&}
			\setcounter{no}{1}
			\renewcommand{\arraystretch}{1.3}
			\setlength{\tabcolsep}{1.3em}
			\begin{tabular}{*{\value{tasks}}{|c}|}
				\hline
				\whiledo{\value{no} < \value{tasks}}{ 		% \exercises
					\theno\and\stepcounter{no}%
				} $\sum$ \\ \hline
				\setcounter{no}{1}
				\whiledo{\value{no} < \value{tasks}}{ 		% \exercises 
					\and\stepcounter{no}%
				} \\[0.5em] \hline
			\end{tabular}
		\end{minipage}
	}
	% Dokumentkopf
	\begin{center}
	{\large \professor}\par
	{\LARGE \textbf{\course, \semester}}\par
	{\large \ifthenelse{\equal{\gender}{f}}{Tutorin}{Tutor}: \tutor, Tutorium \tutorial}\par
	{\Large Übung \exercise}\par 
	{\large \students}\par
	%\date{\AdvanceDate[-1]\today}
	%{\AdvanceDate[-1]\today}
	%\today
	\end{center}
	\vspace{-2ex}
	\rule{\linewidth}{.4pt}
	\vspace{-4ex}
}

% //////////////////// Colors ////////////////////
\let\definecolor=\xdefinecolor
\definecolor{orange}{rgb}{0.8,0.3,0.3} 						% Für Code-Farben notwendig
\definecolor{bgcolour}{rgb}{0.97,0.97,0.97}
\definecolor{codegreen}{rgb}{0,0.6,0}
%\definecolor{codegray}{rgb}{0.35,0.35,0.35}
\definecolor{codepurple}{rgb}{0.58,0,0.82}
\definecolor{codeblue}{rgb}{0.4,0.5,1}

% //////////////////// Settings ////////////////////
%\textheight = 230mm									% Höhe des Satzspiegels / Layouts
%\footskip = 10ex										% Abstand zw. Fußzeile und Grundlinie letzter Textzeile
\parindent 0pt											% verhindert Einrückung der 1. Zeile eines Absatzes
%\setkomafont{sectioning}{\rmfamily\bfseries}				% setzt Ü-Schriften in Serifen, {disposition}

\pgfplotsset{compat=1.14} 								% Funktionsgraphen
\allowdisplaybreaks										% Nadja

\newcommand{\professor}{Wolfgang Mulzer}
\newcommand{\tutor}{Johannes Nixdorf}
\newcommand{\gender}{m}
\newcommand{\tutorial}{4 (Do, 8-10 Uhr)}
\newcommand{\exercise}{06}
\newcommand{\course}{Nichtsequentielle und verteilte Programmierung}
\newcommand{\semester}{SoSe 18}
\newcommand{\students}{Mark Niehues, Anton Oehler}
\newcommand{\taskCount}{3}

% /////////////////////// BEGIN DOKUMENT /////////////////////////
\begin{document}

\head % erstellt die Titelseite

% /////////////////////// Aufgabe 1 /////////////////////////
\section*{1. Semaphore}
\subsection*{a)}
Das Programm ist schreibt abwechselnd \texttt{p} und \texttt{q}, beginnend mit \texttt{p}. Die beiden Prozesse sind so ineinander verschachtelt, dass nur abwechselnde Schreibvorgänge möglich sind. Falls nun \texttt{a1} oder \texttt{b1} entfernt wird, ist der entsprechende Prozess nicht mehr gezwungen auf die Ausführung des letzten Arbeitsschrittes des anderen Prozesses zu warten und wird in unregelmäßigen Abständen schreiben. Der jeweils andere Prozess ist ggf. gezwungen auf die Ausführung des Prozesses ohne 1. Arbeitsschritt zu warten, da dieser immernoch für das Erhöhen des Semaphores zuständig ist.

\subsection*{b)}
Ein \texttt{D.k = 2} wird erreicht, wenn genügend Prozesse in schritt \texttt{a6} warten, also die Semaphore S freigegeben ist, \texttt{k = -2} ist, jedoch \textit{D.P()} in \texttt{a7} nicht ausgeführt wird. Wenn zu diesem Zeitpunkt ein Prozess bevorzugt wird, wird dieser jedes mal, wenn er Abschnitt \texttt{a10} erreicht, den \texttt{count} vergrößern und bei \texttt{a3} verringern. Dieser ist jedoch immer $<= 0$, sodass die Semaphore \texttt{D} erhöht wird.
\begin{table}[h]
\centering
\caption{Szenario bei dem \texttt{D.k = 2} erreicht wird.}
\label{1b}
\begin{tabular}{lllllll}
p1 & p2 & p3 & p4 & S.k & D.k & k \\
(a6, 1)   & (a6, 0)   & (a6, -1)   & (a6, -2)   & 1 & 0 & -2 \\
(\textbf{a8}, 1)   & (a6, 0)   & (a6, -1)   & (a6, -2)   & 1 & 0 & -2 \\
(\textbf{a9}, 1)   & (a6, 0)   & (a6, -1)   & (a6, -2)   & 1 & 0 & -2 \\
(\textbf{a10}, 1)   & (a6, 0)   & (a6, -1)   & (a6, -2)   & \textbf{0} & 0 & -2 \\
(\textbf{a11}, 1)   & (a6, 0)   & (a6, -1)   & (a6, -2)   & 0  & 0 & \textbf{-1} \\
(\textbf{a12}, 1)   & (a6, 0)   & (a6, -1)   & (a6, -2)   & 0 & 0 & -1 \\
(\textbf{a13}, 1)   & (a6, 0)   & (a6, -1)   & (a6, -2)   & 0 & \textbf{1} & -1 \\
(\textbf{a1}, 1)   & (a6, 0)   & (a6, -1)   & (a6, -2)   & 1 & 1 & -1 \\
(\textbf{a2}, 1)   & (a6, 0)   & (a6, -1)   & (a6, -2)   & 1 & 1 & -1 \\
(\textbf{a3}, 1)   & (a6, 0)   & (a6, -1)   & (a6, -2)   & \textbf{0} & 1 & -1 \\
(\textbf{a4}, 1)   & (a6, 0)   & (a6, -1)   & (a6, -2)   & 0 & 1 & \textbf{-2} \\
(\textbf{a5}, \textbf{-2})   & (a6, 0)   & (a6, -1)   & (a6, -2)   & 0 & 1 & -2 \\
(\textbf{a6}, -2)   & (a6, 0)   & (a6, -1)   & (a6, -2)   & \textbf{1} & 1 & -2 \\
(a6, -2)   & (\textbf{a8}, 0)   & (a6, -1)   & (a6, -2)   & 1 & 1 & -2 \\
(a6, -2)   & (\textbf{a9}, 0)   & (a6, -1)   & (a6, -2)   & 1 & 1 & -2 \\
(a6, -2)   & (\textbf{a10}, 0)   & (a6, -1)   & (a6, -2)   & \textbf{0} & 1 & -2 \\
(a6, -2)   & (\textbf{a11}, 0)   & (a6, -1)   & (a6, -2)   & 0 & 1 & \textbf{-1} \\
(a6, -2)   & (\textbf{a12}, 0)   & (a6, -1)   & (a6, -2)   & 0 & 1 & -1 \\
(a6, -2)   & (\textbf{a13}, 0)   & (a6, -1)   & (a6, -2)   & 0 & \textbf{2} & -1 \\
\end{tabular}
\end{table}

\section*{2. Erzeuger-Verbraucher mit einem Ringpuffer}
\subsection*{a)}
Durch die größe des Arrays \texttt{buffer} ist die Buffergröße definiert. Die Variablen \texttt{head} und \text{tail} iterieren in einer Schleife über dieses Array (daher Ringbuffer). Die Semaphoren stellen dabei sicher, dass sich die Zeiger nicht überholen können, sondern stattdessen aufeinander warten.

Falls zum Beispiel der Erzeuger N Elemente beschrieben hat, so sind die Semaphoren im Zustand: \texttt{notEmpty.k = N, notFull.k = 0}. Nun würde im nächsten erzeugenden Schritt \texttt{buffer[0]} überschrieben werden. Der Befehl an Stelle \texttt{p2} stellt allerdings sicher, dass der Erzeugerprozess nun erst warten muss, bis der Verbraucherprozess das Element an Stelle 0 verbraucht hat und in \texttt{c4} die Semaphore \texttt{notFull} wieder erhöht.

\subsection*{b)}
siehe Vorlesung

\subsection*{c)}


\section*{3. Speisende Philosophen}
\section*{4. Kritische Abschnitte mit Semaphoren}

% /////////////////////// END DOKUMENT /////////////////////////
\end{document}
