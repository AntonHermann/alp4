\documentclass[a4paper,twoside,12pt]{article}
%\usepackage [reqno] {amsmath}
\usepackage{amsfonts,amstext}
\usepackage{amsmath}
\usepackage{german}
\usepackage{graphicx}
\usepackage{fullpage}

\newcommand{\ZETTELNUMMER}{1}
\newcommand{\ABGABEDATUM}{am 27. April 2018 bis 10 Uhr in die
Tutorenf\"acher}

\newcounter{AUFGNR}
\setcounter{AUFGNR}{1}
\newcommand{\AUFGABE}[2]{\vspace{0.3cm}\item[Aufgabe~\arabic{AUFGNR}]\stepcounter{AUFGNR} #1\hfill\emph{#2}}


\newcommand{\floor}[1]{\left\lfloor{#1}\right\rfloor}
\newcommand{\ceil}[1]{\left\lceil{#1}\right\rceil}
\newcommand{\half}[1]{\frac{#1}{2}}
\newcommand{\N}{\mathbb{N}}



\renewcommand{\labelenumi}{(\alph{enumi})}
\renewcommand{\labelenumii}{(\roman{enumii})}


\begin{document}
\pagestyle{empty}
\hrule\medskip
\rule{0ex}{0ex}\\[-1ex]
\ZETTELNUMMER. Aufgabenblatt zur Vorlesung

\smallskip
\noindent
\large
\textbf{Nichtsequentielle und Verteilte Programmierung}\hfill SoSe
2018 \\[0.5ex]
\normalsize
Anton Oehler, Mark Niehues

\medskip\hrule

\begin{description}
\AUFGABE{Logik und diskrete Mathematik}{10 Punkte}

\begin{enumerate}
\item \emph{Was ist der Unterschied zwischen einem Booleschen
  Ausdruck und einer Booleschen Funktion?}

  Ein boolescher Ausdruck ist lediglich eine Verkn\"upfung auf eienr unbekannten
  Menge, w\"ahrend eine boolesche Funktion explizit  auf einer booleschen Algebra
  definiert ist.
\item \emph{Sei $(a_i)_{i \in \N}$ eine Folge von nat\"urlichen Zahlen.
  Geben Sie eine pr\"adikatenlogische Formel an, die besagt,
  dass $(a_i)$ unendlich viele gerade Zahlen enth\"alt.
  Begr\"unden Sie Ihre Antwort.}

  \[
    \forall j \in \N \: \exists k,n \in N, k > j: (a_k = 2n)
  \]
  F\"ur jedes $a_j$ gibt es ein Element $a_k$ nach $a_j$, das sich als
  $a_k = 2n$ darstellen l\"asst. Da $n$ ein beliebiges Element aus $\N$ ist,
  folgt daraus, dass $a_k$ eine gerade Zahl ist.
\item \emph{Negieren Sie den folgenden pr\"adikatenlogischen
  Ausdruck:}
    \[
      \forall i \in \N: (i > 5) \Rightarrow (\exists j \in \N: j + i > 100).
    \]
  Negation:
    \[
      \exists i \in \N: (i > 5) \wedge (\forall j \in \N : j + i \leq 100)
    \]
\item \emph{Beweisen Sie durch vollst\"andige Induktion, dass die Potenzmenge
  einer $n$-elementigen Menge genau $2^n$ Elemente enth\"alt, f\"ur alle
  $n \in \N$.}

  \begin{enumerate}
    \item Indutionsbehauptung\\
    Sei $M$ eine beliebige Menge, so gilt:
    \[
      \forall M, |M| = n \in N: |\mathcal{P}(M)| = 2^n
    \]
    \item Induktionsanfang mit $n = 0$\\
    \begin{flalign*}
      M &= \{\}\\
      \Rightarrow \mathcal{P}(M) &= \{\emptyset\}\\
      \Rightarrow |\mathcal{P}(M)| &= 1 = 2^0
    \end{flalign*}
    \item Induktionsschritt\\
    F\"ur jede endliche Menge mit $n$ Elementen gilt $|\mathcal{P}(M)| = 2^n|$\\
    Nun sei $|M| = n+1|$, $M = \{a_0 \dots a_{n+1}\}$.\\
    Sei $U \subseteq M$. Es gibt 2 M\"oglichkeiten:
    \begin{enumerate}
      \item $a_{n+1} \notin U$:\\
      $U$ ist Teilmemge von $M' = M \ \{a_{n+1}\}$ und nach Induktionsvoraussetzung
      gibt es $2^n$ Teilmengen von $M'$
      \item $a_{n+1} \in U$:\\
      $U = U' \cup \{a_{n+1}\}$, $U'$ ist Teilmenge von $M'$.\\
      Nach Voraussetzung gibt es $2^n$ M\"oglichkeiten f\"ur $U'$ und damit gibt
      es $2^n$ Teilmengen von M, die $a_{n+1}$ enthalten:
      \[
        \Rightarrow |\mathcal{P}(M)| = |\mathcal{P}(M')| + |\mathcal{P}(M')| =
        2 \cdot 2^n = 2^{n+1}
      \]
    \end{enumerate}
  \end{enumerate}
\item \emph{Klaus besitzt 10 gelbe Dominosteine, Frank hat 20 rote
  Dominosteine, und G\"unter 15 blaue Dominosteine.
  Die Dominosteine einer Farbe sind jeweils in einer festen Reihenfolge
  nummeriert.
  Auf wie viele verschiedene Arten k\"onnen Klaus, Frank und G\"unter
  ihre Dominosteine hintereinander stellen, so dass die gelben, roten
  und blauen Steine jeweils in der richtigen Reihenfolge bleiben, die
  Farben sich aber beliebig abwechseln d\"urfen?
  Begr\"unden Sie Ihre Antwort.}

  Alle m\"oglichen Kombinationen der 45 Dominosteine ohne Einschr\"ankung: $45!$\\
  Anzahl der ''Dopplungen'' durch Ver\"anderung der Reihenfolge innerhalb einer Farbe
  ist $n_{Farbe}!$, wobei $n_{Farbe}$ die Anzahl der Steine einer Farbe ist.\\
  $\Rightarrow$ Die Steine knnen auf $\frac{45!}{10! \cdot 20! \cdot 15!}
  \approx 1,04 \cdot 10^{19}$ verschiedene Arten angeordnet werden.
\end{enumerate}

\AUFGABE{Zu viel Milch}{10 Punkte}

Waltraud und Manfred wohnen zusammen.
Beide trinken gerne Bio-Milch, die schnell schlecht wird.
Daher sollte immer genau eine Bio-Milch im K\"uhlschrank sein.
Am 10. April 2018 tr\"agt sich folgendes zu:
\begin{verbatim}

        Waltraud	                         Manfred
15:00  Ueberprueft Kuehlschrank:
        keine Milch.
15:05  Geht zum Bioladen.
15:10  Kauft Milch.                      Ueberprueft Kuehlschrank:
                                          keine Milch.
15:15  Verlaesst Bioladen.               Geht zum Bioladen.
15:20  Wieder zu Hause:                  Kauft Milch.
        Stellt Milch in den Kuehlschrank.
15:25                                    Verlaesst Bioladen.
15:30                                    Wieder zu Hause:
                                          Oh nein! Zu viel Milch!
\end{verbatim}

\"Uberlegen Sie sich zwei verschiedene Verfahren, wie Waltraud
und Manfred das Milchproblem mit m\"oglichst einfachen Mitteln
in den Griff bekommen k\"onnen.
Diskutieren Sie die Verfahren bez\"uglich Korrektheit, Effizienz
und Annahmen \"uber die erlaubten Operationen.
Was ist der Zusammenhang zu nichtsequentieller Programmierung?

\begin{itemize}
  \item Wenn man den K\"uhlschrank \"uberpr\"uft und zum Bioladen geht, legt
    man einen Zettel in den K\"uhlschrank, der signalisiert, dass
    man einkaufen ist. \"uberpr\"uft dann die 2. Person den K\"uhlschrank,
    sieht sie dass momentan jemand einkaufen ist und geht selbst nicht
    los.

    \textbf{Korrektheit}: Vorausgesetzt, dass das Einkaufen immer zu Ende ausgeführt wird,
    terminiert dieser Ablauf immer korrekt.\\
    \textbf{Effizienz}: Gut: Es geht niemand unn\"otig los und muss niemals l\"anger auf die Milch
    warten, als er selbst zum Einkaufen brauchen würde.
  \item Rollenverteilung: Nur einer von beiden ist zust\"andig f\"ur das Auff\"ullen
  der Milch.

    \textbf{Korrektheit}: Vorausgesetzt, der Zust\"andige holt verl\"asslich immer die Milch,
    wenn sie alle ist, dann ist auch dieser Ablauf korrekt.\\
    \textbf{Effizienz}: Ineffizient, da, selbst wenn Person B zuerst entdeckt, dass keine
    Milch da ist, aber nicht zust\"andig ist, muss sie warten, bis die zust\"andige Person da ist
    und Milch holt.
\end{itemize}

Das Milchholen entspricht dem kritischen Abschnitt. Es sollten niemals 2 Personen gleichzeitig
Milch kaufen gehen sollten, sonst kann es zu Problemen wie dem vom 10. April kommen.\\
Die Milch ist damit eine geteilte Resource, der Zettel entspricht einem Lock/Flag.

\AUFGABE{Java}{10 Punkte}

Lesen Sie die Java-Dokumentation zu \texttt{Thread}
und \texttt{Runnable}.

Schreiben Sie ein Java-Programm, das als Kommandozeilenparameter eine
nat\"urliche Zahl $n$ erh\"alt und dann $n$ Threads startet.
Jeder dieser Threads soll f\"ur eine zuf\"allige Zeitspanne zwischen
$1$ und $5$ Sekunden laufen und dabei wiederholt seinen eindeutige
Kennung ausgeben.
\end{description}
\end{document}